\documentclass[paper=letter, fontsize=11pt]{scrartcl} % A4 paper and 11pt font size
\newcommand{\horrule}[1]{\rule{\linewidth}{#1}} % Create horizontal rule command with 1 argument of height
\setlength{\parskip}{1em}
\setlength{\parindent}{3em}
\usepackage{listings}

\usepackage{hyperref}
\usepackage{enumerate}
\usepackage{amsmath}
\usepackage{geometry}
 \geometry{
 bottom=1in,
 right=1in,
 left=1in,
 top=1in,
 }

 \usepackage[flushleft]{threeparttable}
\usepackage[capposition=top]{floatrow}

\usepackage{graphicx}
 \graphicspath{ {../figures/} }

\title{	
\normalfont \normalsize 
\horrule{0.5pt} \\[0.4cm] % Thin top horizontal rule
 \large{{\textbf{ECON 293 Homework 2: Commentary}}} \\ % The assignment title
\horrule{2pt} \\[0.5cm] % Thick bottom horizontal rule
}

\author{\small{Jack Blundell, Spring 2017}} % Your name

\date{} % Today's date or a custom date

\begin{document}

\maketitle % Print the title

In this commentary I discuss results and include some key figures. Many further figures and all code are provided in the attached .html file. I worked with Luis Armona on the code. This commentary is written individually.

As in the previous homework, we use data from ``Does Price Matter in Charitable Giving? Evidence from a Large-Scale Natural Field Experiment", by Dean Karlan and John List (AER 2007). In this experiment, two-thirds of recipients of a charity solicitation letter receive some kind of match-donation treatment, whereas the remaining control group receives the letter alone, with no matching promise. We use the same censoring rule as in homework 1 to eliminate observations, emulating an observational study for the first part of this homework.

\section{Observational study}

\subsection{Propensity forest}

Propensity forest. Basically just estimating prop score using a random forest. Then doing standard propensity score matching. 

\subsection{Gradient forest}

See from slide 31 in lecture slides 6b. Get a forest based kernel from averaging tree-based neighborhoods. Local solutions. Data adaptive kernel weights.

\section{Randomized trial}

\subsection{LASSO}

\subsection{Honest causal tree}

\subsection{Causal forest}

\subsection{Causal forest (Gradient forest)}

\end{document}
